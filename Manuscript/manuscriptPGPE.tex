%% ****** Start of file aiptemplate.tex ****** %
%%
%%   This file is part of the files in the distribution of AIP substyles for REVTeX4.
%%   Version 4.1 of 9 October 2009.
%%
%
% This is a template for producing documents for use with 
% the REVTEX 4.1 document class and the AIP substyles.
% 
% Copy this file to another name and then work on that file.
% That way, you always have this original template file to use.

%\documentclass[aip,graphicx]{revtex4-1}
%\documentclass[aip,reprint]{revtex4-1}

%\usepackage{graphicx}

%\draft % marks overfull lines with a black rule on the right
%\documentclass[pre,aps,floatfix,authordate1-4,twocolumn]{revtex4-1}
%\documentclass[pre,aps,floatfix,authordate1-4]{revtex4-1}

\documentclass[aps,prl,superscriptaddress,twocolumn]{revtex4}



%\documentclass[aps,prl,preprint,groupedaddress]{revtex4}

\usepackage{rotating} 
\usepackage{times}
\usepackage{graphicx}
\usepackage{setspace}
\usepackage{amsmath}
\usepackage{epstopdf}
\usepackage[obeyFinal]{easy-todo}
\usepackage{csquotes}

\begin{document}

% Use the \preprint command to place your local institutional report number 
% on the title page in preprint mode.
% Multiple \preprint commands are allowed.
%\preprint{}

\title{NMRlipids IV: Headgroup \& glycerol backbone structures, and cation binding in bilayers with PE and PG lipids} %Title of paper

% repeat the \author .. \affiliation  etc. as needed
% \email, \thanks, \homepage, \altaffiliation all apply to the current author.
% Explanatory text should go in the []'s, 
% actual e-mail address or url should go in the {}'s for \email and \homepage.
% Please use the appropriate macro for the type of information

% \affiliation command applies to all authors since the last \affiliation command. 
% The \affiliation command should follow the other information.

\author{O. H. Samuli Ollila}
\email[]{samuli.ollila@helsinki.fi}
%\homepage[]{Your web page}
\affiliation{Institute of Organic Chemistry and Biochemistry,
Academy of Sciences of the Czech Republic, 
Prague 6, Czech Republic}
\affiliation{Institute of Biotechnology, University of Helsinki}


% Collaboration name, if desired (requires use of superscriptaddress option in \documentclass). 
% \noaffiliation is required (may also be used with the \author command).
%\collaboration{}
%\noaffiliation

\date{\today}

\begin{abstract}
% insert abstract here
  Primarily measured but also simulated NMR order parameters will be collected also for other than phophatidylcholine
  (these are discussed in NMRlipids I) headgroup. The information will be used to understand structural differences between 
  different lipid molecules in bilayers.
\end{abstract}

%\pacs{}% insert suggested PACS numbers in braces on next line

\maketitle %\maketitle must follow title, authors, abstract and \pacs

% Body of paper goes here. Use proper sectioning commands. 
% References should be done using the \cite, \ref, and \label commands


%\label{}
\section{Introduction}

In NMRlipids I and II project we were looking for a MD model
which would correctly reproduce headgroup and glycerol
backbone structures and cation binding for PC lipid bilayers \cite{botan15,catte16}.
Here we extend the same goal for other than PC lipids.
Currently the focus is on PE, PG and PS bilayers and their
mixtures with PC. Experimental data with different amounts of 
added salt is now collected and presented in this manuscript.


\begin{figure}[]
  \centering
  \includegraphics[width=9.0cm]{../Figs/lipids.pdf}
  \caption{\label{lipids}
    Chemical structures and labels for the headgroup carbons.
  }
\end{figure}


Absolute values of experimental order parameters for different lipid headgroups are
collected in Fig. \ref{HGorderParameters}. Signs are measured only for PC as far as I know,
thus only absolute values are used for now.
\begin{figure}[]
  \centering
  \includegraphics[width=9.0cm]{../Figs/HGorderparameters.eps}
  \caption{\label{HGorderParameters}
    Absolute values of order parameters for headgroup and glycerol backbone with different headgroups
    from experiments. POPC values are from \cite{ferreira13}, DOPS from \cite{browning80} contains 0.1M of NaCl,
    POPG from \cite{borle85} contains 10nM PIPES, DPPG from \cite{wohlgemuth80} contains  10mM PIPES and 100mM NaCl,
    DPPE from \cite{seelig76}, E.coliPE and E.coliPG are from \cite{gally81}.
  }
\end{figure}

Based on superficial reading, the conclusions in the literature are roughly \\ 
1) glycerol backbone structures are largely similar irrespectively of the headroup \cite{gally81}, \\
2) glycerol backbone and headgroup structure and behaviour are similar in model membranes and in bacteria \cite{gally81,scherer87,seelig90}, \\
3) headgroup structures are similar in PC, PE and PG lipids, while headgroup is more rigid in PS lipids \cite{wohlgemuth80,buldt81}. \\
Extensive discussion about structural details of PE, PG or PS headgroups do not exists (as far as I know), 
In contrast to PC lipids (see \cite{botan15} and references therein).



Several simulations containing PE, PG and PS lipids have been published \cite{??}, \todo{List should be completed}
however, glycerol backbone and headgroup order parameters are not compared to
the experiments (based on superficial reading of literature).

\section{Methods}
\section{Molecular dynamics simulations}
Molecular dynamics simulation data was collected using open collaboration \cite{botan15}.
The simulations are listed in table \ref{systems}.
\begin{table*}[htb]
  %\begin{sidewaystable*}[!p]
  \centering
  \caption{List of MD simulations. The salt concentrations calculated as 
    [salt]=N$_{\rm c} \times$[water]\,/\,N$_{\rm w}$, where [water]\,=\,55.5~M.
    %   these correspond the concentrations reported in the experiments by Akutsu et al.~\cite{akutsu81}.
    %   The lipid force fields named as in our previous work~\cite{botan15}.
  }\label{systems}
  \begin{minipage}[t]{\textwidth}
    \begin{tabular}{l c c r r r r r r c c}
      %\hline
      % some footnotes are not visible in typeset-MS (pdf)
      lipid/counter-ions & force field for lipids / ions & NaCl (mM) & CaCl$_2$\,(mM) &  \footnote{Number of lipid molecules with largest mole fraction}N$_{\rm l}$   &  \footnote{Number of water molecules}N$_{\rm w}$   & \footnote{Number of additional cations}N$_{\rm c}$  & \footnote{Simulation temperature}T (K)  & \footnote{Total simulation time}t$_{{\rm sim}}$(ns) & \footnote{Time used for analysis}t$_{{\rm anal}}$ (ns) &   \footnote{Reference for simulation files}files\\
      \hline
      DPPE  & Slipids \cite{jambeck12b} &0 & 0           & 288 	& 9386 & 0  & 336  & 200 & 100 & \cite{slipidsDPPEfiles}  \\
      POPE  & Slipids \cite{jambeck12b,??} &0 & 0        & 336	& ?    & 0  & 310  & 200 & 100 & \cite{slipidsPOPEfiles}  \\
      \hline
      POPG/Na$^+$  & CHARMM36 \cite{??} \todoi{Correct citation for CHARMM POPG}        &0 & 0        & ? 			   		& ? & 0  & ?  & ? & ? & \cite{??}
      \todoi{Details to be filled and data to be uploaded in Zenodo by Ollila.}  \\
      POPG/Na$^+$  & Slipids \cite{jambeck13}     &0 & 0        & 288 	& 10664 & 0  & 298  & 250 & 100 & \cite{slipidsPOPGfiles} \\
      \hline
      DPPG/Na$^+$  & Slipids \cite{jambeck13}     &0 & 0        & 288 	& 11232 & 0  & 314  & 200 & 100 & \cite{slipidsDPPGfiles} \\
      DPPG/Na$^+$  & Slipids \cite{jambeck13}     &0 & 0        & 288 	& 11232 & 0  & 298  & 400 & 100 & \cite{slipidsDPPGfilesT298K} \\
    \end{tabular}
  \end{minipage}
  %\end{sidewaystable*} 
\end{table*}


\section{Results and Discussion}

\subsection{PE headgroup and glycerol backbone}
\todo{Add results from this discussion: \url{http://nmrlipids.blogspot.com/2017/03/nmrlipids-iv-headgroup-glycerol.html?showComment=1502290977005#c4964787551006026647}}
Order parameters from Slipids simulations 
and experiments for DPPE are shown in Fig. \ref{HGorderParametersPE}.
Glycerol backbone order parameters in Slipids are off from experiments,
as already observed previously for PC lipids \cite{botan15}.
Order parameter signs for PE are not experimentally measured yet.
For headgroup the signs are set to give best agreement with simulations
and for glycerol to be consistent with experimental signs for PC.
Order parameter for $\beta$ carbon shows apparent agreement
with experiments. However, the sign of beta order parameter is positive,
in contrast to PC where negative sign was measured. Thus, the the beta order
parameter agrees with experiment with the assumption that its sign is opposite
than for PC. This is yet to be confirmed by experiments.
Order parameter for $\alpha$ carbon is too close to zero, even if the sign
would be correct.

\begin{figure}[]
  \centering
  \includegraphics[width=9.0cm]{../Figs/HGorderparametersPE.eps}
  \caption{\label{HGorderParametersPE}
    Order parameters for DPPE headgroup and glycerol
    backbone from simulations with Slipids \cite{??} and experiments
    (DPPE from \cite{seelig76} and E.coliPE from \cite{gally81}).
    Absolute values are shown, because signs are not known experimentally.
  }
  \todo{Experimental signs of the order parameters would be highly useful.} \\
\end{figure}

\subsection{PG headgroup and glycerol backbone}
Comparison between experiments and simulations for PG lipids is shown
in Fig. \ref{HGorderParametersPOPG}. The signs are not yet measured
experimentally. They are set to give the best argeement with experiments.
This would suggest that the $\beta$ order parameter would be positive,
in contrast to PC and PS headgroups, were negative signs were measured.
Even thought the signs turned out to be correct, the tested models would
not give a very good argeement with the experiments.
\begin{figure}[]
  \centering
  \includegraphics[width=9.0cm]{../Figs/HGorderparametersPOPG.eps}
  \caption{\label{HGorderParametersPOPG}
    Order parameters for PG headgroup and glycerol backbone from simulations and experiments without CaCl$_2$ 
    (POPG from \cite{borle85} contains 10mM of PIPES, DPPG from \cite{wohlgemuth80} contains 10mM PIPES and 100mM CaCl,
    E.Coli PG results from \cite{gally81}).
    Signs are not known for experimental data. They are determined to give best agreement
    with simulations. This is not reliable and should be corrected when experimental
    data becomes available.
  }
  \todo{More simulation data for lipids with different headgroups to be collected} \\
  \todo{CHARMM GUI simulation contains only counter ions as potassium.
    All experiments here contain some amount of sodium salt. The best
    ion concentrations for comparison should be figured out.} \\
  \todo{Experimental signs of the order parameters would highly useful.}
\end{figure}


\subsection{Effect of PE, PS and PG on PC headgroup}
, the headgroup 
order parameters increase with the addition of negatively charged PS and PG lipids,
decrease when mixed with positively charged surfactants and
are less affected by the addition of zwitterionic PE lipids or cholesterol.
In addition to the results summarized in Fig. \ref{HGorderparametersPCvsPEPSPGchol},
also mixtures of PC with negatively charged PI, CL, PA, and zwitterionic SM
follow the eletrometer concept \cite{scherer87}.
to mixtures of differently charged
lipids are collected from different experiments in Fig. \ref{HGorderparametersPCvsPEPSPGchol}.

%\section{PC lipid headgroup response to different mixtures in experiments}
As shown in Fig. \ref{HGorderparametersPCvsPEPSPGchol}, order parameters of PC
headgroup behave in various lipid mixtures as expected from the electrometer concept \cite{seelig87, scherer87},
i.e., order parameters increase when anionic lipids are mixed with PC and decrease with cationic
surfactants. The changes with the addition of neutral lipids is significantly smaller.
\begin{figure}[]
  \centering
  \includegraphics[width=8.0cm]{../Figs/HGorderparametersPCvsPEPSPGchol.eps}
  \caption{\label{HGorderparametersPCvsPEPSPGchol}
    PC headgroup order parameters from experiments of mixtures with
    PE, PS, PG and cholesterol \cite{scherer87,scherer89,ferreira13}.
    Signs are determined as discussed in \cite{botan15,ollila16}.
  }
\end{figure}


The headgroup order parameters for PC lipids (POPC and DOPC)
mixed with PE, PS and PG lipids are shown in Fig. \ref{HGorderparametersPCvsPEPSPG}
from different simulation model and experiments \cite{scherer87} with different
mole fractions. As already discussed previosly, the PC lipid headgroup behaviour
follows the electrometer concept in experiments when mixed with other lipids, i.e., the order
parameters increase when mixed with negatively charged lipids (PS, PI, CL, PA and PG)
remains almost unchaged when mixed with neutral lipids (PE and SM) \cite{scherer87}.
This is not the case in simulation data shown in Fig. \ref{HGorderparametersPCvsPEPSPG}.
The addition of DOPE into a POPC and DOPC bilayers significantly decreases the PC headgroup
order parameters in simulations with OPLS compatible version of the Berger force field \cite{tieleman06}
in contrast to experiments \cite{scherer87}. On the other hand, the increase of the PC
headgroup order parameters in CHARMM36 simulations mixed with PS and PG lipids is significantly
smaller than in experiments.
%\begin{figure}[]
%  \centering
%  \includegraphics[width=8.0cm]{../Figs/HGorderparametersPCvsPEPSPGchol.eps}
%  \caption{\label{HGorderparametersPCvsPEPSPGchol}
%    PC headgroup order parameters from experiments of mixtures with
%    PE, PS, PG and cholesterol \cite{scherer87,scherer89,ferreira13}.
%    Signs are determined as discussed in \cite{botan15,ollila16}.
%  }
%\end{figure}

\begin{figure*}[]
  \centering
  \includegraphics[width=16.0cm]{../Figs/HGorderparametersPCvsPEPSPG.eps}
  \caption{\label{HGorderparametersPCvsPEPSPG}
    PC headgroup order parameters from mixtures with PE, PS and PG
    lipids with various mole fractions from different simulation models and experiments \cite{scherer87}.
    Signs are determined as discussed in \cite{botan15,ollila16}.
  }
  \todo{Simulation of CHARMM36 at 298K should be maybe rerun with Gromacs 5.}
\end{figure*}

\subsection{Effect of PC on PS and PG headgroups}
The headgroup order parameters for PS and PG lipid mixtures with PC
having different mole fractions from simulations and experiments \cite{borle85,roux90}
are shown in Fig. \ref{HGorderparametersPSPGvsPC}. The effect of increasing
amount of PC to PS headgroup seems to qualitatively incorrect in CHARMM36 simulations.
The $\beta$-carbon order parameter increases in experiment, but decreases
in simulations with both tested counterions (Na+ and K+). Larger
$\alpha$-carbon order parameter decreases with the addition of PC
in experiment, while the lower remains unchanged. In simulations the
larger increases and the lower decreases.
Interestingly, the $\alpha$-carbon order parameters are closer to experiments
in pure PS system with K+ counterions than with Na+. 
The changes in PG headgroup order parameters are minor in simulations, which
is in line with the only available experiment for the $\beta$-carbon.
\begin{figure}[]
  \centering
  \includegraphics[width=8.0cm]{../Figs/HGorderparametersPGvsPC.eps}
  \caption{\label{HGorderparametersPSPGvsPC}
    PS and PG lipid headgroup order parameters from mixtures with PC
    lipids with various mole fractions from different simulation models and experiments \cite{borle85,roux90}.
    Signs are not yet known experimentally for PG, thus the signs give by simulations are used.
    Signs for PS are measures as described in SI.
  }
  \todo{Some simulations contain potassium as counterions, while some sodium.
    All experiments here contain some amount of sodium salt. The best
    ion concentrations for comparison should be figured out.} \\
  \todo{Why there is difference between CHARMM36 simulation results from POPS:POPC mixture and
  pure POPS? Discussion in https://github.com/NMRLipids/NMRlipidsIVotherHGs/issues/1}
\end{figure}



\subsection{Ca$^{2+}$ binding in bilayers with negatively charged PG and PS lipids}

PC lipid headgroup order parameters can used to measure ion binding
affinity, because their magnitude is linearly proportional
to the amount of bound charge in bilayer \cite{seelig87,catte16}.
This molecular electrometer concept can be used also
for bilayers containing PC lipids mixed with charged lipids \cite{borle85,macdonald87,roux90}.
This is demonstrated in Figs \ref{OrderParametersWithCaCl},
\ref{OrderParametersWithCaClBELOW1M} and \ref{OrderParameterCHANGESWithCaClBELOW1M},
showing order parameters for PC headgroup $\alpha$ and $\beta$ carbons
as a function of CaCl$_2$ concentration in the presence of different amounts of
negatively charged PS or PG lipids.
%\begin{figure}[]
%  \centering
%  \includegraphics[width=9.0cm]{../Figs/LIPIDSwithCaCl.eps}
%  \caption{\label{OrderParametersWithCaCl}
%    PC headgroup order parameters as a function of CaCl concentration from experiments containing charged lipids.
%    Pure DPPC data from \cite{akutsu81}, pure POPC data from \cite{altenbach84}, 
%    POPC:POPS mixture data from \cite{roux90}, POPC:POPG mixture data with 0.1M NaCl from \cite{macdonald87}
%    and POPC:POPG mixture data without NaCl from \cite{borle85}.
%  }
%  \todo{Check the NaCl concentrations in the samples.}
%\end{figure}
%\begin{figure}[]
%  \centering
%  \includegraphics[width=9.0cm]{../Figs/LIPIDSwithCaClBELOW1M.eps}
%  \caption{\label{OrderParametersWithCaClBELOW1M}
%    Figure \ref{OrderParametersWithCaCl} zoomed to smaller concentrations.
%  }
%\end{figure}
%\begin{figure}[]
%  \centering
%  \includegraphics[width=9.0cm]{../Figs/CHANGESwithCaCl.eps}
%  \caption{\label{OrderParameterCHANGESWithCaClBELOW1M}
%    The change of PC headgroup order parameters
%    in the presence of different amount of negatively charged lipids
%    respect to the values without added CaCl$_2$.
%    The original data is the same as in Fig. \ref{OrderParametersWithCaCl}. 
%  }
%\end{figure}

PC headgroup order parameters increase when negatively charged
PS or PG are added to PC bilayer in the absense of added CaCl$_2$,
as expected based on electrometer concept \cite{seelig87}
(see Fig. \ref{OrderParametersWithCaClBELOW1M}).
%In electrometer concept this is explained by the tilting of
%headgroup more parallel to membrane normal \cite{??}.
Further, the order parameters decrease with the addition
of CaCl$_2$ and the decrease is more pronounced for systems with more
negatively charged lipids (see Fig. \ref{OrderParameterCHANGESWithCaClBELOW1M}).
At CaCl$_2$ concentrations ($\sim$ 50-300mM) where order parameters reach the values for pure PC,
the Ca2+ binding presumably fully cancels the charge from negative lipids and
overcharging occurs above these concenterations.
The interpretation of this data and some other results has been that \cite{seelig90}
\begin{displayquote}
  {\it ''(i) Ca$^{2+}$ binds to neutral lipids (phosphatidylcholine, phosphatidylethanolamine) and negatively charged lipids
    (phosphatidylglycerol) with approximately the same binding constant of K = 10-20 M$^{-1}$; \\
    (ii) the free Ca$^{2+}$
    concentration at the membrane interface is distinctly enhanced if the membrane carries a negative surface
    charge, either due to protein or to lipid; \\
    (iii) increased inter-facial Ca$^{2+}$ also means increased amounts
    of bound Ca$^{2+}$ at neutral and charged lipids; \\
    (iv) the actual binding step can be described by a Langmuir
    adsorption isotherm with a 1 lipid:1 Ca$^{2+}$ stoichiometry, provided the interfacial concentration C$_M$, is
    used to describe the chemical binding equilibrium.''}
\end{displayquote}


Comparison of Ca$2+$ binding in PG between CHARMM36 simulations and experiments \cite{borle85}
is shown in Fig. \ref{changesWITHCaClPG}. The decrease of $\alpha$ order parameter
is in agreement with experiments, while decerase of $\beta$ order parameter is overestimated.
The result is very similar to the results with PC in NMRlipids II publication \cite{catte16}.
It should be, however, noted that the $\beta$-order parameters are not actually measured for PG,
but they are calculated from empirical relation $\Delta S_{\beta}=0.43\Delta S_{\alpha}$ \cite{akutsu81}.
Anyway, the data presented in NMRlipids II project and in Fig. \ref{changesWITHCaClPG} together
suggest that Calcium binding is similarly overestimated by CHARMM36 model in pure POPC bilayers and mixtures with
POPG. The good agreement of $\alpha$ carbon would be explained by too weak dependence of its order
parameter of bound charge
\todo{The response of CHARMM36 to cationic surfactant against experiments \cite{scherer89} to be checked.
  I have already ran the simulations, analysis to be done.}.
\begin{figure}[]
  \centering
  \includegraphics[width=9.0cm]{../Figs/CHANGESwithCaClPG.eps}
  \caption{\label{changesWITHCaClPG}
    PG order parameters as a function CaCl$_2$ concentration from experiments \cite{borle85} and CHARMM36 simulations.
    Note that beta order parameter is calculated from empirical relation $\Delta S_{\beta}=0.43\Delta S_{\alpha}$ \cite{akutsu81}, not actually measured. 
  }
\end{figure}

Also dependence of $\beta$-carbon of PG on CaCl$_2$ concentration is compared with
experiments \cite{borle85} in Fig. \ref{PSPGchangesWITHCaCl}. Absolute value of
the order parameter is too large without ions, but rapid decrease due to addition of
CaCl$_2$ is observed in agreement with experiments for systems with 1:1 mixture of POPC and POPG.
In addition, absolute value in systems with CaCl$_2$ is in agreement with experiments.
However, system with 4:1 mixture of POPC and POPG behaves differently, but experimental
data is not available for comparison for this mixture.


\todo{More simulation data for systems with negatively charged lipids and CaCl$_2$ to be collected}

\section{Effect of Ca$2+$ binding on PG and PE headgroup}

Also the experimental order parameters for PS and PG headgroups
as a function of CaCl$_2$ concentration are shown in Fig. \ref{PSPGchangesWITHCaCl}.
\todo{These should be compared to simulations for potential structural interpretation of the changes.}
\begin{figure}[]
  \centering
  \includegraphics[width=9.0cm]{../Figs/PGwithCaCl.eps}
  \caption{\label{PSPGchangesWITHCaCl}
    PG and PS order parameters a function CaCl$_2$ concentration taken from \cite{borle85} and \cite{roux90}, respectively.
  }
  \todo{Get the small concentration data from the inserts}
\end{figure}

\todo{There are mass density profiles available from CHARMM simulations, preliminary figures should be made:
http://nmrlipids.blogspot.com/2017/03/nmrlipids-iv-headgroup-glycerol.html?showComment=1501446779291#c7311317114567929503}


\section{Conclusions}


% Tables may be be put in the text as floats.
% Here is an example of the general form of a table:
% Fill in the caption in the braces of the \caption{} command. Put the label
% that you will use with \ref{} command in the braces of the \label{} command.
% Insert the column specifiers (l, r, c, d, etc.) in the empty braces of the
% \begin{tabular}{} command.
%
% \begin{table}
% \caption{\label{} }
% \begin{tabular}{}
% \end{tabular}
% \end{table}

% If you have acknowledgments, this puts in the proper section head.
\begin{acknowledgments}
% Put your acknowledgments here.
\end{acknowledgments}
\newpage
\appendix
\begin{center}
{\bf SUPPLEMENTARY INFORMATION}
\end{center}



%\section{Measurements of order parameter sign}

%Fig. \ref{PShgSIGNS} summarizes the experimental results on the order parameter sign
%measurement for POPS sample. The experimental protocol is the same used in Ref. \citenum{ferreira16}.
%In (a) you see the headgroup region of the INEPT spectrum where alpha and beta are
%identified. In (b) you have the R-PDLF slices for alpha and beta where you see one single
%splitting for beta (which gives an order parameter equal to 0.12), and for alpha a superposition
%of a large splitting (order parameter equal to 0.09) and a very small splitting which cannot be
%calculated. On the bottom you have the S-DROSS slices of these two carbons. The grey lines show a
%random collection of slices from noise such that it gets clear what is significant. The S-DROSS
%slice for beta clearly shows that the order parameter is negative. The slice for alpha shows that
%the higher order parameter is positive and suggests that the smaller order parameter is negative
%(from the deviation towards negative values in the longer t1 times).
%\begin{figure}[]
%  \centering
%  \includegraphics[width=9.0cm]{../Figs/PShgSIGNS.pdf}
%  \caption{\label{PShgSIGNS}
%    Experimental results for sign measurement for POPS sample
%  }
%\end{figure}

%The results updated with SIMPSON simulations for the SDROSS profiles
%are shown in Fig. \ref{PShgSIGNSsimpson}. The value for the smaller
%alpha order parameter is taken from Fig 3 in Ref. \citenum{roux91},
%because resolution in 13C NMR experiments was nor high enough to determine
%numerical value for this. The plots in Fig. \ref{PShgSIGNSsimpson} (c) show
%the following. The error bars and points are the experimental SDROSS data.
%The thick lines are SIMPSON simulations. The simulations were done by using
%the order parameter for beta equal to -0.12 and for alpha one order parameter
%equal to 0.09 and the other equal to -0.02 (black) or 0.02 (grey).
%Since the black lines agree with experimental data, we conlude that
%the order parameters for $\beta$ carbon are -0.12 and for $\alpha$
%order parameters are 0.09 and -0.02.
%\begin{figure}[]
%  \centering
%  \includegraphics[width=9.0cm]{../Figs/PShgSIGNSsimpson.pdf}
%  \caption{\label{PShgSIGNSsimpson}
%    Experimental results for sign measurement for POPS sample
%  }
%\end{figure}

%\section{Dihedrals}
%\begin{figure*}[]
%  \centering
%  \includegraphics[width=8.0cm]{../Figs/dihed1.png}
%  \includegraphics[width=8.0cm]{../Figs/dihed2.png}
%  \includegraphics[width=8.0cm]{../Figs/dihed3.png}
%  \includegraphics[width=8.0cm]{../Figs/dihed4.png}
%  \includegraphics[width=8.0cm]{../Figs/dihed5.png}
%  \includegraphics[width=8.0cm]{../Figs/dihed6.png}
%  \includegraphics[width=8.0cm]{../Figs/dihed7.png}
%  \includegraphics[width=8.0cm]{../Figs/dihed8.png}
%  \includegraphics[width=8.0cm]{../Figs/dihed9.png}
%  \includegraphics[width=8.0cm]{../Figs/dihed10.png}
%  \caption{\label{dihedrals}
%    Experimental results for sign measurement for POPS sample
%  }
%\end{figure*}

% Create the reference section using BibTe
\bibliography{refs.bib}

%\newpage
%\section{APPENDIX: The NMR results reported by Tiago Ferreira}

\listoftodos

\end{document}
%
% ****** End of file aiptemplate.tex ******
